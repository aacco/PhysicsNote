\documentclass[dvipdfmx,uplatex]{jsarticle}
\usepackage{macros}
\usepackage{siunitx}

\begin{document}
\subsection{分子運動論}
\begin{align*}
\Delta \langle \bm{p} \rangle &= \langle \bm{F} t \rangle \\
&= nN_A \times 2m \langle v_x \rangle \times \frac{\langle v_x\rangle t}{2l} \\
\langle \bm{F} \rangle &= \frac{nN_A m \langle v_x^2 \rangle}{l} \\
P &= \frac{F}{S} = \frac{nN_A m \langle v_x^2 \rangle}{l^3} \\
PV &= nN_a m \langle v_x^2 \rangle \\
&= \frac{1}{3} nN_A m \langle v^2 \rangle \\
&= nRT \\
\end{align*}
これを比較すると
\begin{align*}
nRT &= \frac{2}{3} nN_a \frac{1}{2} m \langle v^2 \rangle \\
\frac{3RT}{2N_A} &= \frac{1}{2} m \langle v^2 \rangle \\
\frac{3}{2}k_bT &= \frac{1}{2} m \langle v^2 \rangle\\
\end{align*}
\begin{defi}[温度Tの定義]
\[
 \frac{3}{2}k_bT = \frac{1}{2} m \langle v^2 \rangle\\
\]
$k_B$はボルツマン定数:
\[
k_B = \frac{R}{N_A} = \SI{1.38e-23}{\joule \per \kelvin}
\]
\end{defi}

\subsection{内部エネルギー}
\begin{defi}[内部エネルギー]
内部エネルギーは系の運動エネルギーと位置エネルギーの和をいう.
特に理想気体では以下.
\[
U = \frac{1}{2}m \langle v^2 \rangle \times N
\]
\end{defi}
\begin{align*}
U &= \frac{1}{2}m \langle v^2 \rangle \times N \\
&= \frac{3}{2}k_BT \times N \\
&= \frac{3}{2}nRT \\
&= \frac{3}{2}pV
\end{align*}
また内部エネルギーは系の温度によって決まる.

\subsection{モル比熱}
\begin{table}[htb]
\begin{center}
	\begin{tabular}{|llll|}
		&Q	&$\Delta U$	&$W_{out}$ \\ \hline
	定積	&$nC_v \Delta T$	&$nC_v \Delta T$&0 \\
	定圧	&$nC_p \Delta T$	&$nC_v \Delta T$&$P \Delta V$\\
	等温	&Q&0&$W^*$\\
	断熱	&0&$nC_v \Delta T$ &$-nC_v \Delta T$\\
	一般	&Q&$nC_v \Delta T$&(P-Vグラフ)
	\end{tabular}
\end{center}
\end{table}
\begin{defi}[モル比熱] \mbox{} \\
$\SI{1}{\mole}$の気体を$\SI{1}{\kelvin}$上昇させるのに要する熱量
\[
C = \frac{Q}{n \Delta T} 
\]
\end{defi}
\begin{theo}[単原子理想気体の定積モル比熱]
\[
C_V = \frac{3}{2}R
\]
\end{theo}
単原子理想気体で定積変化を考えると
\[
Q = \Delta U = nC_V \Delta T
\]
これと
\[
\Delta U = \frac{3}{2}nR \Delta T
\]
を比較して
\[
C_V = \frac{3}{2} R
\]
を得る.(単原子理想気体の定積モル比熱)
これより

\begin{theo}[Mayerの関係式]
\[
C_P =C_V + R
\]
\end{theo}



\end{document}
