\documentclass[dvipdfmx,uplatex]{jsarticle}
\usepackage{macros}
\usepackage{siunitx}
\usepackage{amssymb}


\begin{document}
\subsection{分子運動論}
一辺が$l$の立方体の系を考えると
\begin{align*}
\Delta \langle \bm{p} \rangle &= \langle \bm{F} t \rangle \\
&= nN_A \times 2m \langle v_x \rangle \times \frac{\langle v_x\rangle t}{2l} \\
\therefore \langle \bm{F} \rangle &= \frac{nN_A m \langle v_x^2 \rangle}{l} \\
\end{align*}
ゆえに
\begin{align*}
P &= \frac{\langle F \rangle}{S} = \frac{nN_A m \langle v_x^2 \rangle}{l^3}, \\
\therefore PV &= nN_a m \langle v_x^2 \rangle \\
&= \frac{1}{3} nN_A m \langle v^2 \rangle \\
\end{align*}
これを理想気体の状態方程式(実験則)と比較すると
\begin{align*}
nRT &= \frac{2}{3} nN_A \frac{1}{2} m \langle v^2 \rangle \\
\therefore \frac{3RT}{2N_A} &= \frac{1}{2} m \langle v^2 \rangle \\
\therefore \frac{3}{2}k_BT &= \frac{1}{2} m \langle v^2 \rangle\\
\end{align*}
\begin{defi}[温度Tの定義]
\[
 \frac{3}{2}k_bT = \frac{1}{2} m \langle v^2 \rangle\\
\]
ここで$k_B$はボルツマン定数:
\[
k_B = \frac{R}{N_A} = \SI{1.38e-23}{\joule \per \kelvin}
\]
\end{defi}

\subsection{内部エネルギー}
\begin{defi}[内部エネルギー] \mbox{} \\
系の運動エネルギーと位置エネルギーの和を内部エネルギーという. \\
特に理想気体では以下.
\[
U = \frac{1}{2}m \langle v^2 \rangle \times N
\]
\end{defi}
\begin{cor}[内部エネルギーの表現]
\begin{align*}
U &= \frac{1}{2}m \langle v^2 \rangle \times N \\
&= \frac{3}{2}k_BT \times N \\
&= \frac{3}{2}nRT \\
&= \frac{3}{2}PV
\end{align*}
これによると内部エネルギーは系の温度によって決まり,反応の経路によらない.
\end{cor}

\subsection{モル比熱}
\begin{table}[htb]
\begin{center}
	\begin{tabular}{|l|lll|}
		&Q =	&$\Delta U$	&$+W_{out}$ \\ \hline
	定積変化	&$nC_v \Delta T$	&$nC_v \Delta T$&0 \\
	定圧変化	&$nC_p \Delta T$	&$nC_v \Delta T$&$P \Delta V$\\
	等温変化	&Q&0&$W^*$\\
	断熱変化	&0&$nC_v \Delta T$ &$-nC_v \Delta T$\\
	一般変化	&Q&$nC_v \Delta T$&(P-Vグラフ)
	\end{tabular}
\end{center}
\end{table}
\begin{defi}[モル比熱] \mbox{} \\
$\SI{1}{\mole}$の気体を$\SI{1}{\kelvin}$上昇させるのに要する熱量をモル比熱という.
\[
C = \frac{Q}{n \Delta T} 
\]
\end{defi}
\begin{theo}[単原子理想気体の定積モル比熱]
\[
C_V = \frac{3}{2}R
\]
\end{theo}
単原子理想気体で定積変化を考えると
\[
Q = \Delta U = nC_V \Delta T
\]
これと
\[
\Delta U = \frac{3}{2}nR \Delta T
\]
を比較して
\[
C_V = \frac{3}{2} R
\]
を得る.(単原子理想気体の定積モル比熱)


\begin{theo}[Mayerの関係式]
\[
C_P =C_V + R
\]
\end{theo}
定圧変化において,熱力学第一法則より
\[
nC_p \Delta T = nC_v \Delta T + P \Delta V
\]
ここで$P \Delta V = nR \Delta T$の平衡を保つ反応においては上式から$n \Delta T$を除して
\[
C_p = C_v + R
\]
を得る.


\begin{theo}[Poissonの法則]
平衡を保った断熱変化において,以下が成り立つ.
\begin{align*}
PV^{\gamma} = \mathrm{const.} \\
TV^{\gamma -1} = \mathrm{const.} \\
\frac{P^{\gamma -1}}{T^{\gamma}} = \mathrm{const.} \\
\end{align*}
ここで$\gamma = \frac{C_p}{C_v}$である
\end{theo} 
系の変化:$(P,V,T) \rightarrow (P + \Delta P,V + \Delta V,T + \Delta T)$について
\[
\frac{\Delta P}{P} + \frac{\Delta V}{V} = \frac{\Delta T}{T}
\]
が一次近似的に成り立つ.
\begin{align*}
\because (P + \Delta P)(V + \Delta V) &= nR (T + \Delta T) \\
\end{align*}
また
\[
\frac{dU}{U} = \frac{dT}{T} = - \frac{Rw_{out}}{C_vPV} = -\frac{RdV}{C_vV}
\]
\[
\therefore \int \frac{dT}{T} = - \frac{R}{C_v} \int \frac{dV}{V}
\]
\[
\therefore \log TV^{\frac{R}{C_v}} = \mathrm{const.}
\]
\[
\therefore TV^{\gamma - 1} = \mathrm{const.}
\]
ほかにも
\begin{align*}
 -\frac{RdV}{C_vV} = \frac{dT}{T} &= \frac{d P}{P} + \frac{d V}{V} \\
 \therefore \frac{d P}{P} + \gamma \frac{d V}{V} &= 0 \\
 \therefore PV^{\gamma} &= \mathrm{const.}
\end{align*}

\begin{align*}
 \frac{dT}{T} + \frac{RdV}{C_vV} &= \gamma \frac{dT}{T} - \frac{R}{C_v} \frac{d P}{P} = 0 \\
 \therefore \frac{T^{\gamma}}{P^{\gamma -1}} &= \mathrm{const.}
\end{align*}
のようにして得られる.

\end{document}
