\documentclass[dvipdfmx,uplatex]{jsarticle}
\usepackage{macros}

\begin{document}
\subsection{補遺:力学}
\begin{theo}[場中の運動量保存]\mbox{} \\
一様な重力場中などでの衝突(撃力)の前後の運動量は保存するとしてよい.
\end{theo}
まず力積が
\[
\Delta \bm{p} = m \bm{v}_2 - m \bm{v}_1 = \bm{F} \Delta t
\]
で定義され,特に衝突の力積では$\Delta t \rightarrow 0$と見なせることから,衝突の力:$\bm{F} \rightarrow \infty$が必要で,衝突の力積が$\bm{F} \Delta t$の極限値として得られることがわかる. \\
ここで場による力:$\bm{f}(定数)$による力積は
\[
\Delta \bm{p}_f = \bm{f} \Delta t \to 0
\]
となることから,場の有無にかかわらず衝突前後で運動量は保存する. \\
(この議論での運動量保存とは衝突前後という非常に限られた時間の断片における話であることに注意.)

\subsection{補遺:電磁気学}

\subsection{補遺:熱力学}

\end{document}
