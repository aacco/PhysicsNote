\documentclass[dvipdfmx,uplatex]{jsarticle}
\usepackage{../sty/macros}

\begin{document}
\subsection{補遺:力学}
\begin{theo}[力場中の運動量保存]\mbox{} \\
一様な力場中での衝突(撃力)の前後の運動量は保存するとしてよい.
\end{theo}
まず力積が
\[
\Delta \bm{p} = m \bm{v}_2 - m \bm{v}_1 = \bm{F} \Delta t
\]
で定義され,特に衝突の力積では$\Delta t \rightarrow 0$と見なせることから,衝突の力:$\bm{F} \rightarrow \infty$が必要で,衝突の力積が$\bm{F} \Delta t$の極限値として得られることがわかる. \\
ここで場による力:$\bm{f}(定数)$による力積は
\[
\Delta \bm{p}_f = \bm{f} \Delta t \to 0
\]
となることから,場の有無にかかわらず衝突前後で運動量は保存する. \\
(この議論での運動量保存とは衝突前後という非常に限られた時間の断片における話であることに注意.) \\
詰まる所,$(力場による運動量変化) \ll (撃力による運動量変化)$である.

\begin{theo}[衝突と力学的エネルギー]\mbox{} \\
衝突において完全弾性衝突のとき,力学的エネルギーは保存する.
\end{theo}
\begin{align*}
	v_0 &< V_0 \\
	- \frac{V-v}{V_0 - v_0} &\leq 1 \\
	mv + MV &= mv_0 + MV_0 \\
\end{align*}
を仮定する
	\footnote{
		このとき,$v_0 > V_0$と仮定するとうまくいかない.
		これは
		$-\frac{V-v}{V_0-v_0}\leq 1 
		\leftrightarrow -\frac{v-V}{v_0-V_0}\leq 1$
		で対称性があるように思われるが,
		$v\mapsto V, V\mapsto v$
		で条件が$V+V_0\leq v+v_0$となってしまうことによるようだ.
	}
と
\begin{align*}
	v+v_0 &\leq V+V_0 \\
	m(v - v_0) &= M(V_0 -V)\\
\end{align*}
各辺は正よりこの2式を辺々掛けて
\[
	m(v^2 - v_0^2) \leq M(V_0^2 - V^2) 
\]
ゆえに 
\[
	\frac{1}{2}mv^2 + \frac{1}{2}MV^2 \leq \frac{1}{2}mv_0^2 + \frac{1}{2}MV_0^2
\]
これから反発係数$e$が1のとき力学的エネルギーは保存する. \\
同様にして$e=0$のとき力学的エネルギー変化は
\[
	- \frac{mM}{2(m+M)}(v_0^2 + V_0^2)
\]
となる.

\subsection{補遺:電磁気学}

\subsection{補遺:熱力学}

\end{document}
