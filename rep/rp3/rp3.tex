\documentclass[dvipdfmx,uplatex,a4paper,11pt]{jsarticle}


% 数式
\usepackage{amsmath,amsfonts}
\usepackage{bm}
% 画像
\usepackage[dvipdfmx]{graphicx}


\begin{document}

\title{物理学演習第一 レポート第三回}
\author{nクラス 21B12938 布施太翔}
\date{\today}
\maketitle
\newpage

【1.2.2項 問6】 \\
前回書き忘れていた分を一応自分なりに考えたので書いておきます.\\

c) \\
1.1.1項問5では,
\begin{align*}
  \bm{a}(t) = \bm{a}_0 = Const.
\end{align*}
であるから,
\begin{align*}
  \bm{a}'(t')=\frac{\lambda}{\nu^2}\bm{a}_0
\end{align*}
である.
これは$b)$を満たすので不変性を持つ.\\

d) \\
\begin{align*}
  \bm{f}(\bm{x}(t),\bm{v}(t);t)&=- \omega^2 \bm{x}(t) \\
  \bm{f}(\bm{x}'(t'),\bm{v}'(t');t')&=- \omega^2 \lambda \bm{x}(t)
\end{align*}
である.これは\\
$\nu=1$のとき,$b)$を満たし不変性を持つが,\\
$\nu \neq 1$のとき,$b)$を満たさないので不変性を持たない.\\

\newpage

【1.3.2項 問2】 \\

仮定(1.25)から
\begin{align*}
  \bm{v} = \bm{f}(\bm{x}(t);t)
\end{align*}
と書ける. \\
空間一様性より,
\begin{align*}
  \bm{v} = \bm{f}(t)
\end{align*}
空間等方性より,任意の,$\bm{x}$と同じ次数の正方行列$\mathbb{P}$について
\begin{align*}
  \bm{f}(\bm{x}(t);t) = \mathbb{P}^{-1} \bm{f}(\mathbb{P} \bm{x}(t);t)
\end{align*}
であることにより,
\begin{align*}
  \bm{f}(t) = \mathbb{P}^{-1} \bm{f}(t)
\end{align*}
これが任意の$\mathbb{P}$で成り立つためには
\begin{align*}
  \bm{v} = \bm{f} = \bm{f}(t) = 0
\end{align*}
が必要十分である.これは孤立粒子は静止し続け,等速直線運動は許されないことを示している. \\

【1.3.2項 問2】 \\
 
絶対時間の過程より,
\begin{align*}
  U=U'=0,C=C'=1
\end{align*}
である.
(1.117)より
\begin{align*}
  \left(
  \begin{matrix}
    R & -V \\
    -U & C \\
  \end{matrix}
  \right)^{-1}
  &=
  \left(
  \begin{matrix}
    R' & -V' \\
    -U' & C' \\
  \end{matrix} 
  \right) \\
  \left(
  \begin{matrix}
    R & -V \\
    -U & C \\
  \end{matrix}
  \right)
  \cdot 
  \left(
  \begin{matrix}
    R' & -V' \\
    -U' & C' \\
  \end{matrix}
  \right)
  &=
  \left(
  \begin{matrix}
    RR' & -RV'-V \\
    0 & 1 \\
  \end{matrix}
  \right)
  =
  E_2
\end{align*}
よって
\begin{align*}
  R' = R^{-1}, V' = -R^{-1}V
\end{align*}
である.
ここで,この関係が(1.111)でなく(1.138)の時に成り立つものであることを考えると,
\begin{align*}
  R' = R^{-1} = 1, V' = -R^{-1}V = V
\end{align*}
がわかる.
(1.114),(1.116)と以上より,それぞれ
\begin{align*}
  V &= v - v' \\
  V' &= v' - v \\
  V' &= -V
\end{align*}
を得る.


\end{document}
