\documentclass[a4paper,11pt]{jsarticle}


% 数式
\usepackage{amsmath,amsfonts}
\usepackage{bm}
% 画像
\usepackage[dvipdfmx]{graphicx}


\begin{document}

\title{物理学演習第一 レポート第三回}
\author{nクラス 21B12938 布施太翔}
\date{\today}
\maketitle
\newpage

【1.2.2項 問6】 \\
前回書き忘れていた分を一応自分なりに考えたので書いておきます.\\

c) \\
1.1.1項問5では,
\begin{align*}
  \bm{a}(t) = \bm{a}_0 = Const.
\end{align*}
であるから,
\begin{align*}
  \bm{a}'(t')=\frac{\lambda}{\nu^2}\bm{a}_0
\end{align*}
である.
これは$$b)$$を満たすので不変性を持つ.\\

d) \\
\begin{align*}
  \bm{f}(\bm{x}(t),\bm{v}(t);t)&=- \omega^2 \bm{x}(t)
  \bm{f}(\bm{x}'(t'),\bm{v}'(t');t')&=- \omega^2 \lambda \bm{x}(t)
\end{align*}
である.これは\\
$\nu=1$のとき,$b)$を満たし不変性を持つが,\\
$\nu \neq 1$のとき$b)$を満たさないので不変性を持たない.\\


\end{document}