\documentclass[a4paper,11pt]{jsarticle}


% 数式
\usepackage{amsmath,amsfonts}
\usepackage{bm}
% 画像
\usepackage[dvipdfmx]{graphicx}


\begin{document}

\title{物理学演習第一 レポート第二回}
\author{nクラス 21B12938 布施太翔}
\date{\today}
\maketitle
\newpage

【1.1.2項 問5】 \\

(1.31)では2粒子が相互作用できる,同じ系に存在しているかもしれないことが分かる.\\
(1.35)では2粒子の運動方程式が独立しているので,互いに相互作用しない,独立した系に2粒子があることが分かる.\\
\\

【1.2.2項 問6】 \\

a) \\
スケール変換後の速度と加速度は連鎖律より
\begin{align*}
  \bm{v}'(t') &= \frac{\mathrm{d} \bm{x}'(t')}{\mathrm{d} t'} \\
  &= \frac{\mathrm{d} \lambda \bm{x}(t)}{\mathrm{d}t} \frac{\mathrm{d} t}{\mathrm{d} t'} \\
  &= \frac{\lambda}{\nu} \dot{\bm{x}}(t) \\
  \bm{a}'(t') &= \frac{\mathrm{d} \bm{v}'(t')}{\mathrm{d} t'} \\
  &= \frac{\mathrm{d} \lambda \bm{v}(t)}{\nu \mathrm{d}t} \frac{\mathrm{d} t}{\mathrm{d} t'} \\
  &= \frac{\lambda}{\nu^2} \ddot{\bm{x}}(t) \\
\end{align*}
となる.

b) \\
スケール変換(1.91)の下で運動方程式(1.37)が不変になる条件は
\begin{align*}
  \ddot{\bm{x}}(t) &= \bm{f}(\bm{x}(t), \dot{\bm{x}}(t); t) \\
  \ddot{\bm{x}}'(t') &= \bm{f}(\lambda \bm{x}(t), \frac{\lambda}{\nu} \dot{\bm{x}}(t); \nu t)
  = \frac{\lambda}{\nu^2} \ddot{\bm{x}}(t)
\end{align*}
この2式が恒等であること.よって求める条件は
\begin{align*}
  \frac{\nu^2}{\lambda} \bm{f}(\lambda \bm{x}(t), \frac{\lambda}{\nu} \dot{\bm{x}}(t); \nu t)
  = \bm{f}(\bm{x}(t), \dot{\bm{x}}(t); t)
\end{align*}
である.

\end{document}
