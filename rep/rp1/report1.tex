\documentclass[dvipdfmx,uplatex]{jsarticle}
\usepackage{../sty/macros}

\begin{document}

注:以下では,
\[
	\bm{\psi} = 
		\begin{pmatrix}
		x\\
		y\\
		z
		\end{pmatrix},
	\bm{\varphi} =
		\begin{pmatrix}
		u\\
		v\\
		w
		\end{pmatrix}
\]
とする.(二次元の場合は第2行まで)

\subsubsection{問1}
\[
	\bm{\psi} = 
		\begin{pmatrix}
		x \\
		y	
		\end{pmatrix}
\]
とすると
\[
	^* \bm{\psi} = 
		\begin{pmatrix}
		y\\
		-x	
		\end{pmatrix}
\]
であるが,これらには
\[
	y-xi = (x+yi) \left( \cos \left(-\frac{\pi}{2}\right)+i \sin \left(-\frac{\pi}{2}\right)\right)
\]
の関係がなりたつ.よって題意は示せた.

\subsubsection{問3}
クロス積の定義より,
\[
	\bm{\psi} \times \bm{\psi} = 
		\begin{pmatrix}
		x\\
		y\\
		z
		\end{pmatrix}
	\times
		\begin{pmatrix}
		x\\
		y\\
		z
		\end{pmatrix} =	
		\begin{pmatrix}
		yz-zy\\
		zx-xz\\
		xy-yx
		\end{pmatrix}
		= \bm{0}
\]
よって(A.30)は示せた.
(A.31):
\begin{align*}
	\bm{\varphi} \cdot ( \bm{\varphi} \times \bm{\psi}) &=
		\begin{pmatrix}
		u\\
		v\\
		w
		\end{pmatrix} \cdot
		\begin{pmatrix}
		vz - wy\\
		wx-uz\\
		uy-vx
		\end{pmatrix} \\
		&= u(vz-wy)+v(wx-uz)+w(uy-vx)\\
		&= 0
\end{align*}
よって(A.31)は示せた.

\subsubsection{問6\footnote{参考文献;数研出版;基礎からのチャート式}}
(B.67)の式を$g(x)$,(B.71)の式を$h(x)$とおくと,
\begin{align*}
	g(x) &= \frac{\dff f(x)}{\dff x} \\
	&= \lim_{\varepsilon \rightarrow 0} \frac{f(x+\frac{\varepsilon}{2})-f(x- \frac{\varepsilon}{2})}{\varepsilon} \\
	&= \lim_{\varepsilon \rightarrow 0} \frac{f(x+\frac{\varepsilon}{2})-f(x)+f(x)-f(x- \frac{\varepsilon}{2})}{\frac{\varepsilon}{2}+\frac{\varepsilon}{2}}
\end{align*}
ここで,$\varepsilon \rightarrow 0$ならば$\frac{\varepsilon}{2} \rightarrow 0$であるから
\begin{align*}
	g(x) &= \lim_{\frac{\varepsilon}{2} \rightarrow 0} \left( \frac{f(x+\frac{\varepsilon}{2})-f(x)}{\varepsilon} + \frac{f(x-\frac{\varepsilon}{2})-f(x)}{\varepsilon} \right)
	&= \frac{h(x)}{2} + \frac{h(x)}{2}
	&= h(x)
\end{align*}
よって(B.67)と(B.71)は等しい.

\end{document}
