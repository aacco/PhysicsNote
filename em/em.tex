\documentclass[dvipdfmx,uplatex]{jsarticle}
\usepackage{macros}

\begin{document}
\subsection{Maxwell方程式}
\begin{theo}
\begin{align}
\nabla \cdot \vec{E} &= \dfrac{\rho}{\varepsilon}\\
\nabla \cdot \vec{B} &= 0\\
\nabla \times \vec{E} &= -\dfrac{\partial \vec{B}}{\partial t}\\
\nabla \times \vec{B} &= \mu\varepsilon \dfrac{\partial \vec{E}}{\partial t} + \mu\vec{j}
\end{align}
\end{theo}

\subsubsection{Gaussの法則}
\begin{theo}
\[
\int_S \bm{E} \cdot \bm{n} ds = \frac{Q}{\varepsilon _0}
\]
\end{theo}

\subsubsection{Amp\`{e}reの法則}
\begin{theo}[アンペールの法則] \mbox{}\\
\[
\oint_{\partial S} \bm{H} \cdot d \bm{l} = \int_S \bm{J} \cdot d \bm{S} = I \\
\]
ただし、\\
H	: 磁場の強さ,
J	: 電流密度,
I	: 積分領域 S を貫く総電流,
dl	: 線素ベクトル,
dS	: 面素ベクトル,
$\partial S$	: 面Sの境界 \\
\end{theo}
またこれゆえに
\[
\mathrm{rot} \bm{H} = \nabla \times \bm{H} = \bm{J} \\
\]

\subsubsection{Faradayの電磁誘導の法則}
\begin{theo}
\[
V = -N \frac{\Delta \Phi}{\Delta t} \\
\]
(Nは巻き数)\\
さらに
\[
\oint_S \bm{E} \cdot d\bm{s} = - \frac{d \Phi _B}{dt}\\
\]
\[
\nabla \times \bm{E} = - \frac{\partial \bm{B}}{\partial t}
\]
\end{theo}

\subsection{静電磁場}

\subsubsection{静電ポテンシャル}

\subsubsection{ポアソン方程式}

\subsubsection{定常電流}

\subsubsection{ビオ・サバールの法則}
\begin{theo}[ビオ・サバールの法則] \mbox{} \\
微小な長さの電流要素$I d\bm{l} $によって$ \bm{r} $離れた位置に作られる微小な磁場$ d\bm{H} $は
\[
d\bm{H} = \frac{Id\bm{l} \times \bm{r}}{4 \pi r^3} = \frac{Id\bm{l}}{4 \pi r^2} \times \frac{\bm{r}}{r} \\
\]
\end{theo}

\subsubsection{アンペール力,ローレンツ力}

\subsubsection{コンデンサー}

\subsection{動電磁場}

\subsection{回路}

\subsubsection{キルヒホッフの法則}

\end{document}
